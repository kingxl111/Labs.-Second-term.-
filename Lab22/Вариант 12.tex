\documentclass[a4paper,12pt]{article}
\usepackage[14pt]{extsizes}
\usepackage[english, russian]{babel}
\usepackage[T1]{fontenc}
\begin{document}
\justifying
\setcounter{page}{295}
\noindent где $\epsilon(\Delta y)$ -- непрерывная в нуле функция и
$$ \lim\limits_{\Delta y\to 0} \epsilon(\Delta y) =0 .$$
Разделив обе части равенства (9.22) на $\Delta x$ \neq$ 0, получим 
$$\frac{\Delta z}{\Delta x} =  F^\prime(y_0)\frac{\Delta y}{\Delta x} + \epsilon(\Delta y)\frac{\Delta y}{\Delta x}. \eqno (9.23)$$ $
$ \indent Функция $y = f(x)$ имеет производную в точке $x_0$, т. е. существует предел
$$ \lim\limits_{\Delta x\to 0} \Delta y = 0. $$

\noindent При $ \Delta x = 0 $ имеем $ \Delta y = 0 $. Следовательно, приращение $ \Delta y $, рассматриваемое как функция $ \Delta x $, непрерывано в точке $ \Delta x = 0 $. Поэтому, согласно правилу замены переменных в пределеньных соотношениях, содержащих непрерывные функции (см. п. 5.16), 
$$ \lim\limits_{\Delta y\to 0} \epsilon(\Delta y) = 0. \eqno (9.25)$$
\indent Теперь из (9.23), переходя к пределу при $\Delta$ x $\rightarrow$ 0 в силу (9.24) и (9.25), получим формулу (9.21).
\setmainfont{Century Gothic}
\par \textsc{Замечание 1.} Формула (9.21) для производной сложной функции справедлива и в том случае, 
когда под производными понимаются соответствующие односторонние производные, если только предварительно потребовать, чтобы сложная функция, которая необходима для определения рассматриваемой односторонней (или двусторонней) производной, стоящей в левой части формулы 

\par {\sffamily \textsf{\textbf{\textsc{С\,Л\,Е\,Д\,С\,Т\,В\,И\,Е}}}}} (инвариантность формы первого дифференциала относительно преобразования независимой переменной):
$$ dz = F^\prime(y_0)dy = {\sffamily \textsf{\textsc{Ф}}} ^\prime(x_0)dx.\eqno(9.26)$$
\parВ этой формуле $dy = f^\prime(x)dx $ является дифференциалом функции, а $dx$ -- дифференчиалом независимой переменной.
\newpage

\parТаким образом, дифференциал функции имеет один и тот же вид: произведение производной по некоторой переменной на <<дифференциал этой переменной>> -- независимо от того, является ли эта переменная, в свою очередь, функцией или независимой переменной.
\parДокажем это. Согласно формуле (9.6), $dz = 
{\sffamily \textsf{\textsc{Ф}}} ^\prime(x_0) f^\prime(x_0)dx$, отсюда, применив формулу (9.21) для производной сложной функции, получим $ dz = F^\prime(y_0)f^\prime(x_0)dx.$, но $f^\prime(x_0)dz = dy$, поэтому $dz = F^\prime(y_0)dy$. 

\parФормулу (9.26) можно интерпретировать и несколько иначе, если вспомнить, что дифференциалом функции в точке является функция, линейная относительно дифференциала независимой переменной. Согласно (9.21), дифференциал функции ${\sffamily \textsf{\textsc{Ф}}}(x) = F(f(x))$ имеет вид $ d{\sffamily \textsf{\textsc{Ф}}}(x) = F^\prime(y)f^\prime(x_0)dx$, т. е. является результатом подстановки линейной функции $dy = f^\prime(x_0)dx$, с помощью которой задан дифференциал $df$ (где $y = f(x)$), в линеную функцию $dz = F^\prime(y_0)dy$, задающую дифференциал $dF$ (где $z = F(y)$). Иначе говоря, дифференциал композиции ${\sffamily \textsf{\textsc{Ф}}} = F\circ f$ является композицией дифференциалов $dF$ и $df$:
$$ d(F\circ f) = dF \circ df. $$


\par Отметим, что теорема 5 по индукции распространяется на суперпозицию любого конечного числа функций. Например, для сложной функции вида $z(y(x(y)))$ в случае дифференцируемости функций $z(y)$, $y(x)$ и $x(t)$ в соответствующих точках имеет место формула

$$ \frac{dz}{dt} = \frac{dz}{dy}\:\frac{dy}{dx}\:\frac{dx}{dt}. $$

\par Для обозначения производной $z$ сложной функции $z = z(y), y = y(x)$ употребляют также нижний индекс $x$ или $y$, указывающий, по какой из переменных берется производная, т. е. пишут $z^\prime_{x}$ или $z^\prime_{y}$. Часто для простоты штрих опускают, т. е. вместо $z^\prime_{x}$ пишут просто  $z_{x}$. В этих обозначениях формула (9.21) имеет вид $z_{x} = z_{y}y_{x}$.

\par {\sffamily \textsf{\textbf{\textsc{П\,р\,и\,м\,е\,р\,ы. 1.}}}} Пусть $y = x^a, x > 0$, найдем $\frac{dy}{dx}$. Имеем $x^a = e^u$, где $u = a\ln{x}$. Замечая, что $\frac{du}{dx} = \frac{u}{x}$, получаем

$$ \frac{dx^a}{dx} = \frac{de^u}{dx} = \frac{de^u}{du}\:\frac{du}{dx} = e^u \frac{a}{x} = e^{a\ln{x}}\,\frac{a}{x} = ax^{a - 1}.$$





\end{document}
